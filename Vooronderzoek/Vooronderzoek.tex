\documentclass[10pt,a4paper]{article}
\usepackage[utf8]{inputenc}
\usepackage[dutch]{babel}
\usepackage{amsmath}
\usepackage{amsfonts}
\usepackage{amssymb}
\usepackage{graphicx}
\title{Robotica: Vooronderzoek}
\author{I. van Alphen, S. van Doesburg, E.  Salsbach, M. Visser}
\begin{document}
\maketitle

\tableofcontents

\section{Samenvatting}

\newpage

\section{Inleiding}
In deze documentatie word er vooronderzoek gedaan naar de opdracht A van de minor Robotica. Opdracht A betreft het ontwikkelen van de besturing en simulatie van een hexapod robot. Het voornaamste doel van de opdracht is om een koppeling maken tussen een simulatie model en de hardware van de hexapod. Daarnaast word onderzocht of er nog extra functionaliteiten kunnen worden toegevoegd aan de robot.

Probleemstelling
In de huidige situatie wordt de robot handmatig met een afstandbediening bestuurd en kent geen vorm van intelligentie. 

Als voorbereiding op het werken met kunstmatige intelligentie op de robot, is er voor gekozen om een simulatiemodel voor en van de robot te cree\"eren. 

De reden hiervan is dat er in een model oneindig veel verschillende situaties voor de robot gecree\"erd kunnen worden. Bovendien kan het vanuit financie\"el oogpunt in situaties nuttig zijn om niet met de echte hardware te werken. 

Door een koppeling te maken tussen het simulatiemodel en de hardware van de robot...

\begin{itemize}
\setlength\itemsep{0em}
\item Aanleiding 
\item Probleemstelling (kort) 
\item Doelstelling (kort) 
\item Vraagstelling (kort) 
\item Methode 
\item Uitleg opbouw van verslag
\end{itemize}

\section{Opdrachtdefinitie}
\begin{itemize}
\setlength\itemsep{0em}
\item Context van het praktijkprobleem 
\item Relevantie van het ontwerp
\item Probleemstelling (uitgebreid) 
\item Doelstelling (uitgebreid) 
\item Vraagstelling (uitgebreid)
\end{itemize}

\newpage

\section{Theoretisch kader}
\begin{itemize}
\setlength\itemsep{0em}
\item Inhoudelijke verkenning, kennis benodigd voordat met het ontwerp gestart kan worden (o.a. normen en regelgeving)
\item Relevante onderzoeksvragen worden hierin uitgewerkt 
\item Welke literatuur en/of theorieën zijn relevant en wat betekent dit voor het ontwerp \item Overzicht van bestaande oplossingen van het probleem en waarom voldoen deze in dit specifieke geval wel/niet.
\end{itemize}
\newpage

\section{Ontwerp}



\section{Resultaten}


\section{Discussie en resultaten}

\section{Conclusies}

\section{Conclusies}

\section{Aanbevelingen}




\section{Firmware robot}
Om de robot te besturen is er in het standaard model gebruik gemaakt van de Arbotix robocontroller. 
testtest

Er is voor gekozen om niet de standaard software te gebruiken, maar vanaf de grond af aan eigen software te schrijven.

\section{Simulatie robot}
gelukt

\section{Hard- en software koppeling}


\section{Geen boven en onderkant}

\section{Interacteren geluid}

\section{Uitwijk systeem}

\section{Machine learning}

\section{Specificaties}
Must have
\begin{itemize}
\setlength\itemsep{0em}
\item Simulatiemodel robot
\item Firmware robot
\item 
\end{itemize}
Should have
Could have
Won't Have


\section{Bibliografie}
\bibliography{references}
\bibliographystyle{IEEEtran}


\end{document}
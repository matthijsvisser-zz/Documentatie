\documentclass[10pt,a4paper]{article}
\usepackage[utf8]{inputenc}
\usepackage[dutch]{babel}
\usepackage{amsmath}
\usepackage{amsfonts}
\usepackage{amssymb}
\usepackage{graphicx}
\title{Robotica: Vooronderzoek}
\author{I. van Alphen, S. van Doesburg, E.  Salsbach, M. Visser}
\begin{document}
\maketitle
\newpage

\tableofcontents
\newpage

\iffalse % Commentaar sectie
\section{Samenvatting}
Not in use yet
\newpage
\fi

\section{Inleiding}
In deze documentatie wordt er vooronderzoek gedaan naar de opdracht A van de minor Robotica. Opdracht A betreft het ontwikkelen van de besturing en simulatie van een hexapod robot.\cite{beroepsopdrachten}

In de huidige situatie wordt een hexapod handmatig met een afstandbediening bestuurd en kent geen vorm van intelligentie. Ter voorbereiding op het werken met kunstmatige intelligentie, is er voor gekozen om een simulatiemodel van de robot te ontwikkelen. Om praktische informatie te verzamelen is er een koppeling nodig tussen het simulatie model en de hardware van de robot. Met behulp van deze koppeling kan er onderzoek gedaan worden naar bijvoorbeeld effici\"ente looppatronen en zelf lerende functies.

Het verslag is opgebouwd uit het onderzoek naar een hexapod met daarbij de hoofdvraag en deelvragen. Gevolgd door de specificaties en de implementatie van het ontwerp.

\newpage

\section{Opdrachtdefinitie}
De hexapod is momenteel alleen te besturen met behulp van een afstandbediening. De huidige besturingssoftware op de robot is niet in staat naast de afstandbediening om externe commando's te verwerken. Daarnaast kent het in de huidige toestand geen vorm van intelligentie. Zo heeft de robot momenteel geen besef van wat zich in zijn directe omgeving bevindt.

Als voorbereiding op het werken met kunstmatige intelligentie op de robot, is er voor gekozen om een simulatiemodel voor en van de robot te ontwikkelen. De reden hiervan is dat er in een model oneindig veel verschillende situaties voor de robot gecree\"erd kunnen worden. Bovendien kan het vanuit financie\"el oogpunt in situaties nuttig zijn om niet met de echte hardware van de robot te werken. 
Door een real-time koppeling te maken tussen het simulatiemodel en de hardware van de robot is het mogelijk in staat om meer informatie te verzamelen om de intelligentie te verbeteren.

Het uiteindelijke doel is om in een simulatieomgeving de bewegingseigenschappen van de robot te optimaliseren. Om deze software vervolgens te testen op het echte model. Er moet dus onderzocht worden hoe een ontwikkelingsomgeving opgezet kan worden waarbij er een koppeling is tussen een simulatie en de robot zelf. Daarnaast moet de robot extern te besturen zijn met behulp van een computer. De robot en het simulatiemodel moeten de mogelijkheid hebben om de stand van de servomotoren real-time naar elkaar over te brengen. Om beschadiging te voorkomen moet het in staat kunnen zijn om fouten te detecteren. De onderlinge poten zouden niet met elkaar in contact moeten komen. Het onderzoeken van het gedrag van een hexapod is van belang om uiteindelijk de hexapod zelf lerende functies te geven.
\newpage

\iffalse %commentaar sectie
\textbf{Hoofdvraag\\}
De hoofdvraag van het onderzoek betreft:
Hoe wordt een ontwikkelingsomgeving opgezet waarbij een robot aan een simulatie gekoppeld wordt?

\textbf{Deelvragen\\}
De onderzoeksvraag is onder te verdelen in verschillende deelvragen:
\begin{itemize}
\setlength\itemsep{0em}
\item Wat is een hexapod?
\item Welke draadloze techniek kan het beste worden toegepast als communicatiemiddel?
\item Wat zijn effici\"ente looppatronen voor een hexapod op verschillende oppervlakken?
\item Is een hexapod in staat om zich voort te bewegen met één of meerdere beperkingen aan zijn poten?
\item Zijn er effici\"entere looppatronen bij een zwaardere belasting van de hexapod?
\item Zijn er verschillen tussen de simulatie en de werkelijkheid?
\item Hoe detecteert de hexapod dat er een onmogelijke bewegingsactie uitgevoerd moet worden?
\item Hoe verkent de hexapod zijn omgeving en onderscheid deze objecten van elkaar?
\item Kan de robot zich voortbewegen ongeacht de orie\"entatie?
\end{itemize}
\fi

\iffalse % commentaar sectie

\section{Theoretisch kader}
\begin{itemize}
\setlength\itemsep{0em}
\item Inhoudelijke verkenning, kennis benodigd voordat met het ontwerp gestart kan worden (o.a. normen en regelgeving)
\item Relevante onderzoeksvragen worden hierin uitgewerkt 
\item Welke literatuur en/of theorieën zijn relevant en wat betekent dit voor het ontwerp \item Overzicht van bestaande oplossingen van het probleem en waarom voldoen deze in dit specifieke geval wel/niet.
\end{itemize}
\newpage

\section{Ontwerp}
\newpage

\section{Resultaten}
\section{Discussie en resultaten}
\section{Conclusies}
\section{Aanbevelingen}

\section{Firmware robot}
Om de robot te besturen is er in het standaard model gebruik gemaakt van de Arbotix robocontroller. 
Er is voor gekozen om niet de standaard software te gebruiken, maar vanaf de grond af aan eigen software te schrijven.

\section{Simulatie robot}
\section{Hard- en software koppeling}
\section{Geen boven en onderkant}
\section{Interacteren geluid}
\section{Uitwijk systeem}
\section{Machine learning}

\fi

\section{Specificaties}

Voor de specificaties van dit project, is het van belang een onderscheid te maken tussen functionaliteiten die noodzakelijk of gewenst zijn bij het ontwerp. De noodzakelijke functies moeten in ieder geval ge\"implementeerd worden, terwijl het overige optioneel is, afhankelijk van de tijdrestrictie.

\subsection{Noodzakeljke specificaties}
Het primair doel van dit project is om een verbinding te cre\"eren tussen een fysieke robot hexapod en een simulatiemodel. Hierbij moet de hexapod aangestuurd kunnen worden door het simulatiemodel in het programma VREP. Veranderingen aan de stand van de poten moeten direct terug te zien zijn in het simulatiemodel. Daarnaast moet het extern te besturen zijn met behulp van een computer. De verbinding tussen hexapod en de computer dient draadloos te zijn ten gunste van de bewegingsvrijheid van de robot. %(Vind deze zin niet nodig ^M)De fysieke hexapod wordt beschikbaar gesteld door de Hogeschool van Amsterdam, maar het is echter wel nog nodig om de spin te modelleren.
Om te voorkomen dat de hexapod zichzelf kan beschadigen is het noodzakelijk dat de maximale bewegingsvrijheid van de gewrichten (per situatie) wordt uitgerekend of ingesteld. Zodat de poten onderling niet met elkaar botsen of dat de bekabeling beschadigd raakt.

\subsection{Gewenste specificaties}
Er zijn een veel mogelijkheden wat betreft additionele functies die ge\"implementeerd kunnen worden. In deze subsectie zijn een aantal functionaliteiten opgesomd die mogelijk ge\"implementeerd kunnen worden, maar die niet noodzakelijk zijn voor het uiteindelijke eindproduct.

\begin{itemize}
\setlength\itemsep{0em}
\item De hexapod is er zich van bewust als hij ondersteboven is geplaatst, en kan de stand van zijn poten daarop aanpassen. Wanneer de hexapod horizontaal gepositioneerd wordt, dan zorgt dit systeem er voor dat alle poten op of richting de ondergrond zijn geplaatst.
\item Is in staat om zijn poten in- en uit te strekken, zodat het eenvoudig opgeborgen en opgezet kan worden.
\item De spin kan muren en/of objecten detecteren en zijn looproute hier op aanpassen. 
\item Met een beperking aan een of meerdere poten is het in staat om het standaard looppatroon aan te passen.
\item Kan zijn looppatroon aanpassen indien nodig, afhankelijk van het gewicht van de eventuele ballast.
\end{itemize}

\newpage

\section{Bibliografie}
\bibliography{references}
\bibliographystyle{IEEEtran}


\end{document}
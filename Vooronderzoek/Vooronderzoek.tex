\documentclass[10pt,a4paper]{article}
\usepackage[utf8]{inputenc}
\usepackage[dutch]{babel}
\usepackage{amsmath}
\usepackage{amsfonts}
\usepackage{amssymb}
\usepackage{graphicx}
\title{Robotica: Vooronderzoek}
\author{I. van Alphen, S. van Doesburg, E.  Salsbach, M. Visser}
\begin{document}
\maketitle

\section{Inleiding}
In deze documentatie word er onderzoek gedaan naar de opdracht van opdracht A van de minor Robotica. Opdracht A betreft de besturing en simulatie van een hexapod robot. Het voornaamste doel van de opdracht is om een koppeling maken tussen een simulatie model. Daarnaast...

\section{Firmware robot}
Om de robot te besturen is er in het standaard model gebruik gemaakt van de Arbotix robocontroller. 

Er is voor gekozen om niet de standaard software te gebruiken, maar vanaf de grond af aan eigen software te schrijven.

\section{Simulatie robot}
gelukt

\section{Hard- en software koppeling}


\section{Geen boven en onderkant}

\section{Interacteren geluid}

\section{Uitwijk systeem}

\section{Machine learning}

\end{document}
\documentclass[10pt,a4paper]{article}
\usepackage[utf8]{inputenc}
\usepackage[dutch]{babel}
\usepackage{amsmath}
\usepackage{amsfonts}
\usepackage{amssymb}
\usepackage{graphicx}
\title{Robotica: Vooronderzoek}
\author{I. van Alphen, S. van Doesburg, E.  Salsbach, M. Visser}
\begin{document}
\maketitle

\section{Inleiding}
In deze documentatie word er onderzoek gedaan naar de opdracht van opdracht A van de minor Robotica. Opdracht A betreft de besturing en simulatie van een hexapod robot. Het voornaamste doel van de opdracht is om een koppeling maken tussen een simulatie model. Daarnaast...

\section{Onderzoeks opzet}
Het project bestaat uit twee fases. De eerste fase heeft als doel de hexapot in de simulatie te verwerken en ervoor te zorgen dat er een verbinding is tussen de simulatie en de hexapot. 

In de tweede fase zal er gekeken worden naar het schrijven en implementeren van zelflerende functies. Deze fase wordt gestart rond het begin van blok twee, dit vanwege de vakken die worden gegeven in dit blok.

In het begin van fase een moet er gekeken worden naar welke software pakketten geschikt zijn voor de toepassingen waarvoor de hexapot gebruikt gaat worden. Het softwarepakket waarin de simulatie gedaan wordt moet beschikken over de mogelijkheid om te communiceren met de componenten van de hexapot. Hierbij moet er informatie verstuurd en ontvangen kunnen worden.

Daarnaast zal er in de eerste fase gekeken worden naar toepassingen die andere hebben bedacht voor de hexapot. Hieruit is wellicht inspiratie te halen voor handige toepassingen.

Om de verbinding te maken tussen robot en computer dienen er verschillende bronnen te worden geraadpleegd. Een groot deel van die bronnen zal worden voorzien door gebruik te maken van het internet. Verder zal kennis die in de les wordt opgedaan, worden toegepast en de boeken kunnen worden bekeken.

In fase twee zullen vooral de lessen dienen als leidraad voor het ontwikkelen van de zelflerende functies.

De hexapot is al fisiek aanwezig, wat het testen van geschreven code gemakkelijk maakt voor testen. Op deze manier wordt gelijk duidelijk of het geschreven programma functioneerd. Tijdens het schrijven van de  code zal de hexapot daarom aanwezig zijn.



\section{Firmware robot}
Om de robot te besturen is er in het standaard model gebruik gemaakt van de Arbotix robocontroller. 
testtest

Er is voor gekozen om niet de standaard software te gebruiken, maar vanaf de grond af aan eigen software te schrijven.

\section{Simulatie robot}
gelukt

\section{Hard- en software koppeling}


\section{Geen boven en onderkant}

\section{Interacteren geluid}

\section{Uitwijk systeem}

\section{Machine learning}

\end{document}